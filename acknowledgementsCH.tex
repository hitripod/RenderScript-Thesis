\begin{acknowledgementsCH}

\setlength{\baselineskip}{1.5em}
%經過兩年的長跑,攻讀碩士的生涯也終要邁向終點。

%首先我要感謝我的指導教授廖世偉博士,在百忙之中還是不厭其煩的和我討論。他亦師亦友的指導,總讓我學到比所想要的還要多。%
%我要感謝五位口試委員,洪士灝教授、郭大維教授、施吉昇教授、楊佳玲教授以及蔡欣穆教授,因為有他們的批評與指教,這篇論文%
%才能更臻完美。謝謝實驗室的所有同學們,給予我鼓勵和即時的協助。最後,我要感謝我的父母,因為他們的悉心照料讓我能夠%
%無後顧之憂、全心全力的投入於研究。
%
%而在兩年的長跑後,要迎向的,是我人生另一個新的起點。

在這碩士兩年來,最先要感謝的就是我的指導教授廖世偉老師。雖然工作繁重,但總是會盡可能地抽空給予我們協助,%
對我而言,指導教授不像是大家所俗稱的老闆,反而比較像是工作上的夥伴、工作外的朋友。除了課業上的指導外,也關心我們的生涯發展,謝謝老師。\\%

接著要感謝的就是在碩一給我極大鼓勵與幫助的張家榮學長,不論我問什麼問題,總是不厭其煩地耐心指導我。也因為有著老師與學長為榜樣,%
讓我在碩士生涯中,永遠有個需要努力追求的目標。\\%

還要謝謝三位口試委員:蘇雅韻教授、楊佳玲教授、以及黃世勳教授能撥空參加我的口試,也因為有三位口試委員以及指導教授的指點之下,%
才得以完成這篇論文。\\%

最後,感謝父母、家人的全力支持,讓我衣食無憂地將心思全力投注在研究上;感謝實驗室的同學、學弟們,研究課題上的討論讓我學習到許多,而課業外%
因有著你們的陪伴也讓我研究所生活多采多姿。\\%

結束了十八年的學涯後,期許自己能為我們的社會、國家貢獻一些力量。%
\end{acknowledgementsCH}
