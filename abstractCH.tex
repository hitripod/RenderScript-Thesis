\begin{abstractCH}

\setlength{\baselineskip}{1.5em}
為了追求 (1) 效能; (2) 可移植性; (3) 使用性,Google 在 Android 3.0 的版本中推出了 RenderScript。我們延伸其功能,使之可遠端化。具體一點的說法,我們透過網路接口傳送 RenderScript 命令到遠端圖形引擎上。請注意,在此所提到的遠端引擎可以是在同一裝置或是遠端裝置之上。

我們的貢獻分為兩個部分:\\
        \\首先,我們讓 RenderScript 現有的先進先出佇列可以連網。我們建構了一個轉運層以實現網路功能抽象化。\\
        \\第二,透過把指令傳送到遠端引擎之上,我們可以重新在該引擎上執行所接收而來的命令。視不同型態的遠端引擎而定,重新執行命令的動作可能會需要把 RenderScript 的命令對應到 egl、glx、agl、或是 wgl。其分別對應於不同的平台,依序為 Android, Linux, Mac OS, Windows。在本論文中,我們將敘述對應到 egl 的實作細節。

\end{abstractCH}
